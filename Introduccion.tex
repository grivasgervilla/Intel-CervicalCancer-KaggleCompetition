\section{Introducción}

En esta práctica se va a trabajar con el dataset de Intel & MobileODT Cervical Cancer Screening. Este dataset está formado por un cojunto de imágenes de úteros con cáncer cervical. Contiene imágenes de 3 tipos distintos de cáncer, estos tipos son llamados: tipo 1, tipo 2 y tipo 3. Por lo que nos encontramos ante un problema de clasificación de multiclases, a diferecia de la práctica anterior, en la cual solo debíamos predecir si el pasajero moría o sobrevivía. El objetivo es predecir, dada una imagen, que tipo de cáncer presenta.\\


En este trabajo se va a trabajar con el dataset del Titanic, un conjunto de datos en el que se plantea un \textbf{problema de clasificación}, dadas una serie de características de un pasajero, que enumeraremos en la siguiente sección, se tendrá que decidir si el pasajero sobrevivió o no a la catastrofe. En primer lugar, como toma de contacto con el dataset y también para tomar algunas ideas que aplicar al conjunto de datos, se han realizado dos tutoriales con planteamientos similares, ambos realizan un preprocesamiento parecido a los datos y emplean Random Forest como algoritmo final de clasificación, después de probar otros modelos como puede ser uno basado en un árbol de decisión o simplemente suponer que todas las mujeres sobrevivieron y que todos los hombres perecieron. La realización de estos dos tutoriales se adjunta en esta memoria a modo de apéndices.\\

\section{Exploración de datos} \emph{Ver archivo \archive{exploracion.Rmd}}\\

En primer lugar vamos a presentar el conjunto de datos que tenemos, disponemos de 891 instancias en el conjunto de entrenamiento y 418 en el conjunto de test. Estas instancias presentan el siguiente conjunto de atributos:

\begin{enumerate}
\item \textbf{PassengerId: } identificador del pasajero.
\item \textbf{Pclass: } la clase en la que embarcó.
\item \textbf{Name:} el nombre del pasajero.
\item \textbf{Sex: } sexo del pasajero.
\item \textbf{Age: } edad del pasajero.
\item \textbf{SibSp: } número de hermanos/esposos/esposas del pasajero que viajaban también a bordo.
\item \textbf{Parch: } número de padres/hijos del pasajero que viajaban también a bordo.
\item \textbf{Ticket: } número o identificador del ticket de embarque del pasajero.
\item \textbf{Fare: } lo que pagó el pasajero por su pasaje.
\item \textbf{Cabin: } camarote(s) en el(los) que viajó el pasajero.
\item \textbf{Embarked: } puerto desde el que embarcó. C = Cherbourg, Q = Queenstown, S = Southampton.
\item \textbf{Survived: } el pasajero murió (0) o sobrevivió (1). La variable a predecir y que por tanto no está presente en las intancias del conjunto de test.
\end{enumerate}

Lo primero que hemos hecho ha sido estudiar si contábamos con valores perdidos. Gracias a los tutoriales notamos que hay algunos atributos que, si bien no presentan valores perdido (\textbf{NA}), contienen una cadena vacía, lo que se podría considerar también un valor perdido, y por lo tanto esto también se tendrá en cuenta en la fase de preprocesamiento. Tras este análisis vemos que la mayoría de valores perdidos están en el atributo \textbf{Age}, también hay algún valor perdido en el \textbf{Embarked} y muchos valores perdidos para el atributo \textbf{Cabin} (señalar aquí que en R es mejor emplear los operandos condiciones \code{\&} y \code{|}, en lugar de \texttt{\&\&} o \texttt{||}, ya que con estos obtenemos resultados incorrectos). Para estudiar de forma general el conjunto de datos de los que disponemos han sido de utilidad los métodos \code{summary} y \code{str}.\\

De hecho en ambos conjuntos, para el atributo \textbf{Cabin} tenemos aproximadamente un 78\% de valores perdidos, lo cual es una cantidad muy elevada. En el tutorial sobre ingeniería de características que se facilita desde la propia página de la competición se habla de obtener, procesando este campo, la cubierta en la que se alojó el pasajero. Sin embargo, puesto que se trata de un atributo con tantos valores perdidos se ha decidido no realizar dicho preprocesamiento: no tenemos información suficiente para decidir en qué cubierta viajó el pasajero y una imputación, usando por ejemplo el paquete \code{mice}, no sería de mucha confianza al tener un número tan elevado de valores perdidos.\\

Vamos a ver ahora la primera gráfica sobre nuestro conjunto de datos, lo que vamos a reflejar en esta gráfica es el desequilibrio entre las dos clases (549 muertos y 342 sobrevivientes), aunque no se trata de un desequilibrio demasiado grande lo trataremos en la fase de preprocesamiento y trataremos de comparar distintas técnicas de balanceo de clases según su rendimiento para un algoritmo determinado:

\begin{figure}[H]
  \centering
  \includegraphics[width=\textwidth]{imgs/imbalanced.pdf}
  \caption{Desbalanceo entre clases}
\end{figure}

En la gráfica anterior también mostramos en qué proporción sobreviven hombres y mujeres en el conjunto de entrenamiento. Como podemos ver son las mujeres las que en su mayoría sobrevivieron mientras que los hombres tuvieron menos posibilidades, recordemos que durante el accidente del Titanic se estableció la política de evacuar a mujeres y niños en primer lugar. Este hecho nos lleva, en el tutorial de Trevor Stephens, a plantear el modelo sexista.

\begin{figure}[H]
  \centering
  \includegraphics[width=\textwidth]{imgs/classes.pdf}
  \caption{Distribución de sobrevivientes según la clase}
\end{figure}

En esta gráfica lo que vemos es cómo se distribuyen los sobrevivientes en las distintas clases de los pasajeros. Como vemos son los de tercera clase los que perecieron en mayor proporción. Sin embargo observamos algo interesante, y es que en los sobrevivientes lo parece haber tanta diferencia entre las clases, siendo los pasajeros de segunda clase y no los de tercera los que murieron en mayor proporción. Como vemos el análisis exploratorio ha sido principalemente enfocado a ver cómo influyen las distintas características de los pasajeros en su probabilidad de sobrevivir, al fin y al cabo este es el problema que se nos plantea con este dataset.\\

\begin{figure}[H]
  \centering
  \includegraphics[width=\textwidth]{imgs/age.pdf}
  \caption{Distribución de sobrevivientes según la edad y el sexo.}
\end{figure}

Aquí lo que podemos apreciar es en primer lugar que las mujeres sobreviven en mayor proporción que los hombres, por otro lado los hombres jóvenes parecen sobrevivir en mayor proporción que los mayores, al contrario que ocurre con las mujeres.

\begin{figure}[H]
  \centering
  \includegraphics[width=\textwidth]{imgs/fare.pdf}
  \caption{Distribución de sobrevivientes según la edad y el sexo.}
\end{figure}

En esta gráfica apreciamos algo parecido a lo que teníamos en la gráfica con las clases de los pasajeros. Los pasajeros que pagaron menos por su pasaje, los de menor clase suponemos, son los que mueren en mayor proporción. Claro hay muerte que no se corresponden con esto, pensemos en que aquí estamos analizando los factores uno por uno por separado. Hay muchos factores distintos que pudieron influir en la muerte de un pasajero, entre otros su localización en el barco. Como ya hemos dicho este no es un factor que nosotros hayamos tenido en cuenta, y queda como trabajo futuro.\\

Por último vamos a ver cómo se distribuye la tasa de muertos según el tamaño de la familia. Parece lógico pensar que se intentó evacuar a las familias juntas, o que estas presionaron para que se evacuasen a todos sus miembros. Sin embargo, familias con un tamaño muy grande es más complicado que se pudiesen evacuar y quizás, por tal de permanecer unidas perecieran. Por tanto esta variable que se crea nueva, el tamaño de la familia, puede resultar muy interesante:

\begin{figure}[H]
  \centering
  \includegraphics[width=\textwidth]{imgs/fsize.pdf}
  \caption{Distribución de sobrevivientes según el tamaño de su familia.}
\end{figure}

Como podemos ver aquellos pasajeros que viajaban solos tenían mayor probabilidad de morir y por otro lado, las familias con un tamaño demasiado grande, cinco miembros o más, también lo tenían más difícil para sobrevivir.

